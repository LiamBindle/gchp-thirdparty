\documentclass[english]{article}
\usepackage{epsf}
\usepackage{babel}
\usepackage{longtable}
\usepackage{times}
\usepackage{graphicx}
\usepackage{hyperref}
\usepackage[T1]{fontenc}
\usepackage[landscape]{geometry} 

\setlength{\unitlength}{1truecm}
\setlength{\hoffset}{-1.0in}
\setlength{\voffset}{-0.5in}
\setlength{\textwidth}{8.75in}
\setlength{\textheight}{7.0in}

\parindent 0pt

\begin{document}

\begin{titlepage}
\begin{center}
\hspace{1in} \\
\vspace{1in}
{\Large ESMF Test Harness} \\
{\Large Automated Test Suite Report} \\
\medskip
{\it Gary Block} \\
\vspace{.5in}
{\large \today}
\end{center}

%\begin{latexonly}
\vspace{4.0in}
\begin{tabular}{p{5in}p{.9in}}
\hrulefill \\
\noindent http://www.earthsystemmodeling.org \\
\end{tabular}
%\end{latexonly}

\end{titlepage}

% toc needs file from previous pass
\tableofcontents

\newpage

%\noindent
\section{Test Harness Suite Report}
This document summarizes the setup of each individual test case in a test harness test suite.  
The intent is to allow a review of the tests being performed by a test suite. 
This document can be view as two separate documents, one summarizing the redistribution aspects
of a test suite and one summarizing the data initialization and regridding aspects of a test suite.
In addition, each class has its own section for easier referencing.  \\

A test suite is defined through a top level configuration file.  
This file contains references to a set of test descriptions.  
Each test description has a set of problem descriptor strings.
A problem description string has a transformation, a distribution description file,
and a grid description file. \\

A distribution description file has a set of distributions definitions.
A distribution definition describes how a source distribution is transformed 
to a destination distribution. \\

A grid description file has a set of grids definitions.
A grid definition has the size of the grid, the coordinate range, the grid type, and grid 
units of measure for each dimension. \\

Test cases are generated by transforming each grid on each distribution. In the case of a non-gridded class,
the grid information can be used to initialize a distributed array with values. \\

\subsection{Producing a Report}
The test harness produces an XML file summarizing the setup of test cases within a test suite.
The XML file is processed by the program xsltproc using a special latex generation script.  
Since xsltproc normally produces html output, the intermediate latex output is postprocessed to reduce
html escape sequences and insert latex escape sequences where needed.  
The latex output is used to generate the report in PDF format. \\

This document is generated by running gmake from within the \$ESMF\_DIR/src/testharness/catalog
directory.  The test harness is part of the ESMF unit tests and scales the redistribution based
on the number of available processors. The report is dependent on the number of processors
and should allocate the same number of processors that is to be used for the actual test run. \\

Prior to generating the catalog, build the ESMF library and unit tests to make the test harness and
latex source postprocessor. \\


\input{bodyAS}
\input{bodyFS}
\input{bodyA}
\input{bodyF}
\end{document}
  

