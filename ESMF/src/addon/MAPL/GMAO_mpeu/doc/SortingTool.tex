% +-======-+ 
%  Copyright (c) 2003-2007 United States Government as represented by 
%  the Admistrator of the National Aeronautics and Space Administration.  
%  All Rights Reserved.
%  
%  THIS OPEN  SOURCE  AGREEMENT  ("AGREEMENT") DEFINES  THE  RIGHTS  OF USE,
%  REPRODUCTION,  DISTRIBUTION,  MODIFICATION AND REDISTRIBUTION OF CERTAIN 
%  COMPUTER SOFTWARE ORIGINALLY RELEASED BY THE UNITED STATES GOVERNMENT AS 
%  REPRESENTED BY THE GOVERNMENT AGENCY LISTED BELOW ("GOVERNMENT AGENCY").  
%  THE UNITED STATES GOVERNMENT, AS REPRESENTED BY GOVERNMENT AGENCY, IS AN 
%  INTENDED  THIRD-PARTY  BENEFICIARY  OF  ALL  SUBSEQUENT DISTRIBUTIONS OR 
%  REDISTRIBUTIONS  OF THE  SUBJECT  SOFTWARE.  ANYONE WHO USES, REPRODUCES, 
%  DISTRIBUTES, MODIFIES  OR REDISTRIBUTES THE SUBJECT SOFTWARE, AS DEFINED 
%  HEREIN, OR ANY PART THEREOF,  IS,  BY THAT ACTION, ACCEPTING IN FULL THE 
%  RESPONSIBILITIES AND OBLIGATIONS CONTAINED IN THIS AGREEMENT.
%  
%  Government Agency: National Aeronautics and Space Administration
%  Government Agency Original Software Designation: GSC-15354-1
%  Government Agency Original Software Title:  GEOS-5 GCM Modeling Software
%  User Registration Requested.  Please Visit http://opensource.gsfc.nasa.gov
%  Government Agency Point of Contact for Original Software:  
%  			Dale Hithon, SRA Assistant, (301) 286-2691
%  
% +-======-+ 
\section{Sorting Tool}
%
This section presents the modules {\tt m\_MergeSorts} and
{\tt m\_SortingTools}.

\noindent
The module {\tt m\_MergeSorts} contains basic sorting procedures, that in
addition to a couple of standard Fortran 90 statements in the
array syntex, allows a full range sort or unsort operations.
The main characteristics of the sorting algorithm used in this
module are, a) stable, and b) index sorting.

It has two subroutines: {\tt IndexSet} that initializes an array of data
location indices and {\tt IndexSort} that is stable merge index sorting
integers, reals and characters.
%
\begin{verbatim}
EXAMPLES 1:
  Code:
      use m_MergeSorts,only : IndexSet
      use m_MergeSorts,only : IndexSort
      ...
      integer, intent(in) :: No
      type(Observations), dimension(No), intent(inout) :: obs
      integer, dimension(No) :: indx  ! automatic array
      call IndexSet(No,indx)
      call IndexSort(No,indx,obs(1:No)%lev,descend=.false.)
      call IndexSort(No,indx,obs(1:No)%lon,descend=.false.)
      call IndexSort(No,indx,obs(1:No)%lat,descend=.false.)
      call IndexSort(No,indx,obs(1:No)%kt, descend=.false.)
      call IndexSort(No,indx,obs(1:No)%ks, descend=.false.)
      call IndexSort(No,indx,obs(1:No)%kx, descend=.false.)
      call IndexSort(No,indx,obs(1:No)%kr, descend=.false.)
                ! Sorting
      obs(1:No) = obs( (/ (indx(i),i=1,No) /) )
      ...
                ! Unsorting
      obs( (/ (indx(i),i=1,No) /) ) = obs(1:No)
\end{verbatim}
 
\noindent
The module {\tt m\_SortingTools} contains a collection of sorting utilities.  
The utilities are accessed through three generic interfaces, IndexSet(),
IndexSort(), and IndexBin().

Note that, a version of IndexBin() for real arguments is not implemented 
due to the difficulty of comparing two real values as being equal.  
For example, a bin for real values may be specified as a single number, 
a range of two numbers, a number with an absolute error-bar, or a number 
with a relative error-bar.
 
In general, one may have to map both keys(:) and bins(:) to integer indices 
by the a given rule, then use the integer version of IndexBin() with the 
two integer index arrays to do the sorting.
This mapping rule, however, is application dependent.
 
Also note that, in principle, it is possible to use both IndexSort() and 
IndexBin() in the same sorting task.
%
\begin{verbatim}
EXAMPLES 2:
  - An example of using IndexSet()/IndexSort() in combination with
    the convenience of the Fortran 90 array syntex can be found in the
    prolog of m_MergeSorts.
  - An example of using IndexSet()/IndexBin(): Copying all "good"
    data to another array.

  Code:
      integer :: indx(n)
      call IndexSet(n,indx)
      call IndexBin(n,indx,allObs(:)%qcflag,GOOD,ln0=ln_GOOD)
                ! Copy all "good" data to another array
      goodObs(1:ln_GOOD)=allObs( indx(1:ln_GOOD) )
                ! Refill all "good" data back to their original places
      allObs( indx(1:ln_GOOD) ) = goodObs(1:ln_GOOD)

  - Similarily, multiple keys may be used in an IndexBin() call
    to selectively sort the data.  The following code will move data
    with kt = kt_Us,kt_U,kt_Vs,kt_V up to the front:

  Code:
      call IndexBin(n,indx,allObs(:)%kt,(/kt_Us,kt_U,kt_Vs,kt_V/))
      allObs(1:n) = allObs( indx(1:n) )

  - Additional applications can also be implemented with other argument 
    combinations.
\end{verbatim} 
%

