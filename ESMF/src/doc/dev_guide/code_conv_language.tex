% CVS $Id$

\subsection{Lang: Interlanguage Coding Conventions}

ESMF is written in a combination of C/C++ and Fortran.
Techniques used in ESMF for interfacing C/C++ and Fortran codes
are described in the {\it ESMF Implementation Report}\cite{bib:ESMFimplrep},
which is available via the {\bf Users} tab on the ESMF website.  These
techniques, which address issues of memory allocation, passing
objects across language boundaries, handling optional arguments,
and so on, are general and have been applied to multiple projects.  

We distinguish between these techniques and the conventions used
by the ESMF project when interfacing C/C++ and Fortran.  These
conventions, which represent specific implementation choices,
require additional input and explanation, and this section in 
the {\it Guide} is currently incomplete.  The list below outlines
the topics that we intend to address:

\begin{enumerate}
\item Logicals across language interfaces
\item Optional arguments across language interfaces
\item Layering Fortran on top of C/C++
\item Layering C/C++ on top of Fortran
\end{enumerate}


\subsubsection{Optional Arguments Across Language Interfaces}

It is often necessary for C++ code to call Fortran code where optional arguments are
used.  By convention, most Fortran compilers use a C NULL for the argument address
to indicate that the optional argument is missing.

The following Fortran subroutine has optional arguments.  Note that Fortran 77 style
adjustable size array dimensioning is used for array b:

\begin{verbatim}
    subroutine f_ESMF_SomeProcedure (a, b, b_size, rc)
      implicit none
      real,    intent(in),  optional :: a
      integer, intent(in)            :: b_size
      real,    intent(out), optional :: b(b_size)
      integer, intent(out), optional :: rc

      if (present (a)) then
        ... = a
      end if
      ...
      if (present (b)) then
        b(i) = ...
      end if
      ...
      if (present (rc)) then
        rc = localrc
      end if
    end subroutine f_ESMF_SomeProcedure
\end{verbatim}

When calling the above from C++, NULL can be used to indicate missing arguments:

\begin{verbatim}
    extern "C" {
        FTN_X(f_esmf_someprocedure)(float &a, float &b, int &b_size, int &rc);
    }
    ...
    // array b not present, so use NULL.
    int b_size = 0;
    FTN_X(f_esmf_someprocedure)(&a, NULL, &b_size, &rc);
    ...
    // array b is present.  Pass size and address.
    float *b = new float[20];
    int b_size = 20;
    FTN_X(f_esmf_someprocedure)(&a, b, &b_size, &rc);
\end{verbatim}

\subsection{Lang: Fortran Coding Standard}
\label{sec:code_conv_cam}

This standard is derived from the \htmladdnormallink{GFDL Flexible
  Modeling System Developers'
  Manual}{http://www.gfdl.gov/\~{}vb/FMSManual} and the
\htmladdnormallink{coding standard for the NCAR Community Atmospheric
  Model(CAM)}{http://www.cesm.ucar.edu}
developed by Jim Rosinski. Other documents containing coding
conventions include the "Report on Column Physics Standards"
(\htmladdnormallink{http://nsipp.gsfc.nasa.gov/infra/}{http://nsipp.gsfc.nasa.gov/infra})
and "European Standards For Writing and Documenting Exchangeable
Fortran 90 Code"
(\htmladdnormallink{http://nsipp.gsfc.nasa.gov/infra/eurorules.html}{http://nsipp.gsfc.nasa.gov/infra/eurorules.html}).

The conventions assume the use of embedded documentation extractor
\htmladdnormallink{ProTeX}{http://dao.gsfc.nasa.gov/software/protex}.

\subsubsection{Content Rules}

\begin{description}

\item[F95 Standard] ESMF will adhere to the Fortran 95 language
  standard~\cite{ref:f95}, to the extent that it is implemented.

\begin{enumerate}
\item All elements of the ANSI \texttt{f95} standard are permitted,
  with a few listed exceptions whose use is discouraged or prohibited.
  These are enumerated below.
\item Language extensions are severely restricted. They may be used in
  limited fashion, provided a pressing reason exists (e.g major
  performance enhancement using a particular proprietary software
  system), \emph{and} an alternate formulation is provided for compiling
  environments that do not permit the extension.
\item The standard may change in the future, e.g to Fortran 2003, or
  any other, after review.
\end{enumerate}
  
\item[Preprocessing] The use of preprocessing directives is intended
  for language extensions, and in some circumstances, it is used to
  generate module procedures under a generic interface for variables
  of different type, kind and rank (thus circumventing \texttt{f90}'s
  strict typing), while maintaining a single copy of the source.

The use of preprocessor directives in ESMF is permitted under the
following conditions:

\begin{enumerate}
\item Where language extensions are used, \texttt{cpp}
  \texttt{\#ifdef} statements must be used to shield lines from
  compilers that may not recognize them.
\item Use is restricted to the built-in preprocessor of the
  \texttt{f90} compiler (based on \texttt{cpp}), and cannot be based
  on external preprocessors such as \texttt{m4}. This condition may be
  relaxed on platforms where the builtin preprocessor proves to be
  inadequate.
\item Use is restricted to short code sections (a useful rule of thumb
  is that an \texttt{\#ifdef} and the matching \texttt{\#endif} should
  both be visible on a single 80\latexhtml{$\times$}{x}24 editor
  window).
\item Tokens must be uppercase.
\item Owing to restrictions in certain compilers, preprocessor
  variable names may not exceed 31 characters.
\end{enumerate}

\item[Source files] Each source code file defines a single
  \texttt{program} or \texttt{f90 module}. The filename must be the
  same as the module name with the following extensions:
\begin{verbatim}

filename.f - fixed format, no preprocessing.
filename.F - free format, no preprocessing.


filename.f90 - fixed format, with preprocessing.
filename.F90 - free format, with preprocessing.

\end{verbatim}
\item[Module name] The names of Fortran procedure interfaces will be
  preceded by {\tt ESMF\_}.  Class names will normally be the first
  item in a procedure name, followed by the specific method, e.g.,
  {\tt ESMF\_TimeMgrGetCurrDate}. Compilers produce object code for
  each source file, usually with a \texttt{.o} extension. During
  linking, it is required that each object file have a unique name;
  extremely generic names must be avoided. This convention is used to
  prevent name collisions.

\item[Scope] Each module in ESMF must have \texttt{private} scope by
  default.  Each public interface therefore needs to be explicitly
  published.
  
\item[Typing] The use of implicit typing is forbidden. Every module
  must contain the line:

\begin{verbatim}
implicit none
\end{verbatim}

in the module header, and every variable explicitly declared.

There are a few restrictions on the length of a character variable:

\begin{enumerate}
\item Character variables that are \emph{arguments} to routines should
  be declared with \texttt{(len=*)}. It has been observed that
  compilers are inconsistent in their ``padding'' practices, and the
  standard is silent on the subject.
\item It is recommended that other character variables be declared
  with length a multiple of 4, or preferably 8. This is a
  \emph{requirement} for variables that are components of derived
  types, since it has been observed that without these restriction,
  there are occasional word alignment fault errors generated.
\end{enumerate}

\item[Arguments] The \texttt{intent} of arguments to subroutines and
  functions must be explictly specified.
  
\item[Intrinsics] The \texttt{f90} language provides a number of
  intrinsic functions for performing common operations. The use of the
  standard intrinsics is generally encouraged. Notes:

\begin{enumerate}
\item The generic form of the intrinsic (e.g \texttt{max()}) must be
  used rather than the specific one (e.g \texttt{dmax0()}). This
  permits flexibility to later changes of type.
\item Many of the intrinsic array operations have been found to be
  poorly optimized for performance (e.g \texttt{reshape()},
  \texttt{matmul()}) since they have to be perfectly general. These
  must be used with care in code regions that are critical for
  performance.
\item Several older standard intrinsic names have been declared
  obsolescent, and the current names are preferred (e.g
  \texttt{modulo()} instead of \texttt{mod()}, \texttt{real()} instead
  of \texttt{float()}).
\end{enumerate}
  
\item[Constants and magic numbers] Shared constants
  must never be hardcoded: instead mnemonically useful names are
  required. This applies to physical constants such as the universal
  gas constant, gravity, and so on, but also for flags used to select
  code options. In particular, this coding construct:

\begin{verbatim}
subroutine advection(flag)
integer, intent(in) :: flag
...
if( flag.EQ.1 )then
    call upwind_advection( ... )
else if( flag.EQ.2 )then
    call smolar_advection( ... )
...
endif
end subroutine advection
...
call advection(1)
\end{verbatim}

is discouraged. This should instead be written as:

\begin{verbatim}
integer, parameter :: UPWIND=1, SMOLAR=2
...
subroutine advection(flag)
integer, intent(in) :: flag
...
if( flag.EQ.UPWIND )then
    call upwind_advection( ... )
else if( flag.EQ.SMOLAR )then
    call smolar_advection( ... )
...
endif
end subroutine advection
...
call advection(UPWIND)
\end{verbatim}

\item[Procedural interfaces] Procedural interfaces are the public
  interfaces to subroutines and functions provided by a module.

\begin{enumerate}
\item Procedures that perform the same function on different datatypes
  (e.g of differing type, kind or rank) should have a single generic
  interface. When the generic public interface exists, all the
  module procedures that constitute it must be private.
\item Optional arguments, if any, should \emph{follow} the required
  arguments, so that the procedure may be called without explicit
  argument keywords.  Optional arguments in all new public interfaces \emph{must}
  have a \texttt{keywordEnforcer} argument separating the required arguments from the
  optional arguments.  This requires the caller to specify keywords for the arguments,
  allowing for upward compatibility as new optional arguments are added over time.
\item Argument lists should be as short as possible. If necessary, related elements of
  an argument list should be encapsulated in a public derived type.
\end{enumerate}

\item[Deprecated elements of the standard] Deprecated language
  elements include:

\begin{enumerate}
\item implicit typing.  Use \texttt{implicit none} in all modules and external
  procedures.
\item \texttt{common} blocks. Use module global variables instead;
\item assumed size arrays: i.e declarations of the form \texttt{a(*)}
  or \texttt{a(1)} with the intention of over-indexing. This can
  inhibit effective bounds-checking at compile- and runtime.
\item \texttt{STOP} statements: this can generate single-processor
  exits in some parallel environments;
\item array syntax. Though compact and concise, many compilers have
  trouble generating efficient code from source written in this
  notation.
\end{enumerate}

\item[mkmf] The
  \htmladdnormallink{mkmf}{http://www.gfdl.noaa.gov/\~{}vb/mkmf.html} tool
  may be used to generate Makefiles with correct dependencies for F90
  and hybrid-language codes. This places minor restrictions on
  \texttt{module} and \texttt{use} statements: these declarations must
  be on a single line, and the use of continuation lines, e.g:


\begin{verbatim}
module &
  module_name
use &
  module_name
\end{verbatim}

  is forbidden.
  
\item[Use statement]
\begin{enumerate}

\item The \texttt{use} statement must appear on the same line as
  the module name, i.e, do not use:

\begin{verbatim}
use &
  module_name
\end{verbatim}

  This is to be consistent with the dependency analysis performed by
  \texttt{mkmf} outlined above.
\item The \texttt{use, only:} clause is \emph{required} so that all
  imported elements are explicitly declared.
\item Variables imported by a \texttt{use} statement must not be
  modified by the importing module.
\item Modules cannot publish variables and interfaces imported from
  another module. Thus, each public element of a module is only
  available through that module. 
\end{enumerate}

\end{description}

\subsubsection{Style Rules}

Style is somewhat personal, and it would be needlessly restrictive to
attempt to impose style requirements. These are recommendations which
we believe will lead to pleasant encounters with clear, legible and
understandable code. The only style requirement we place is that of
{\it consistency}: a single code unit is required to be rigorous in
using the author's preferred set of stylistic attributes.  It is not
onerous to follow a style: modern editors have many language-aware
features designed to produce a consistent, customizable style.

Style recommendations include the following:

\begin{enumerate}
\item The use of free format;
\item The use of do...end do constructs (as opposed to numbered loops
  as in Fortran-IV);
\item The use of proper indentation of loops and blocks;
\item The liberal use of blank lines to delimit code blocks;
\item The use of comment lines of dashes or dots to delimit
  procedures;
\item The use of useful descriptive names for physically meaningful
  variables; short conventional names for iterators (e.g
  \texttt{(i,j,k)} for spatial grid indices);
\item The use of uppercase for constants (parameters), lowercase for
  variables;
\item The use of verbose syntax on \texttt{end} statements (e.g
  \texttt{subroutine~sub...end~subroutine~sub} rather than
  \texttt{subroutine~sub...end});
\item The use of short comments on the same line to identify
  variables; longer comments in well-delineated blocks to describe
  what a portion of code is doing;
\item Compact code units: long procedures should be split up if
  possible. 200 lines is a rule-of-thumb procedure length limit.
\end{enumerate}

\subsection{Lang: C/C++ Coding Standard}

\begin{itemize}
\item {\bf Use of namespaces}
To avoid conflict in symbols, ESMF C++ code will utilize namespaces to
add either ESMC (interface code) or ESMCI to symbol names.

Example: 

The header file for an {\it ESMC\_Widget} is as follows:
\begin{verbatim}
**file ESMCI_Widget.h
// Internal Widget esmf class
#ifndef ESMCI_Widget_h
#define ESMCI_Widget_h

namespace ESMCI {

class Widget {
public:
Widget();
~Widget();
void Manufacture();
};

} // namespace ESMCI
\end{verbatim}

The implementation file is:
\begin{verbatim}
***file ESMCI_Widget.C
#include <ESMCI_Widget.h>

namespace ESMCI {

Widget::Widget() {
...
}
Widget::~Widget() {
...
}
void Widget::Manufacture() {
...
}

} // namespace ESMCI
\end{verbatim}

Lastly, when this object is used from C interface code, the following constructs are used:

\begin{verbatim}
*** file ESMC_SomeInterface_F.C
#include <ESMCI_Widget.h>

extern "C" {
// this cannot be in the ESMCI namespace, else it would change the
// linkage. We have two choices: 1) add using namespace ESMCI and
// then use Widget, else 2) qualify Widget, ESMCI::Widget
void FTN(c_esmc_someinterface)(
ESMCI::Widget **wptr, int *rc){
#undef ESMC_METHOD
#define ESMC_METHOD "c_esmc_someinterface()"
if (rc!=NULL)
*rc = ESMC_RC_NOT_IMPL;

.....
(*wptr)->Manufacture();

}

} // extern "C"
\end{verbatim}

\item {\bf Protex Prologues}  Each C or C++ function, subroutine, or 
module will include a prologue instrumented for use with the
  \htmladdnormallink{ProTeX auto-documentation script}
{http://dao.gsfc.nasa.gov/software/protex}.  Prologue templates are
included in the ESMF document templates package described in 
Section~\ref{sec:code_templates}.  This convention may be relaxed
when absorbing large external code bases that may have their own
documentation conventions.

\end{itemize}






